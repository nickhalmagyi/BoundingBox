
\documentclass[preprint,12pt]{article}

\usepackage{hyperref, amsmath,amssymb,array, amscd, comment, color, datetime}

%\usepackage{color}
%\usepackage[
%      colorlinks=true,
%%      linkcolor=darkblue,  
%      urlcolor=blue,    
%      filecolor=blue,     
%      citecolor=red,
%      pdfstartview=FitV,
%       bookmarksopen=true    
%      ]{hyperref}
%\usepackage[left=2.5cm,top=2.5cm,right=2.5cm,nohead]{geometry}



%%%%%%%%%%%%%%%%%%%%%%%%%%%%%%%%%%%%%%%%%%%%%%%%%%%%%%%%%%%%%%%%%
\input{/Users/Halmagyi/Dropbox/macro}
%%%%%%%%%%%%%%%%%%%%%%%%%%%%%%%%%%%%
\begin{document}
%%%%%%%%%%%%%%%%%%%%%%%%%%%%%%%%%%%%

\begin{center}
\fbox{DRAFT: \today,\ \currenttime}
\end{center}

\vspace{0.5cm}
\begin{center}
\baselineskip=13pt {\LARGE \bf{Notes on Bounding Box Construction \\
for Filtering Geographic Data}\\}
 \vskip1.5cm 
Nick Halmagyi\ 
 \vskip0.5cm
\textit{Laboratoire de Physique Th\'eorique et Hautes Energies,\\
Universit\'e Pierre et Marie Curie, CNRS UMR 7589, \\
F-75252 Paris Cedex 05, France}\\
\vskip0.5cm
halmagyi@lpthe.jussieu.fr \\
\end{center}

%%%%%%%%%%%%%%%%%%%%%%%%%%%%%%%%%%%%
\section{Introduction}
%%%%%%%%%%%%%%%%%%%%%%%%%%%%%%%%%%%%
A common problem is data-science is the following: given a single geographic location $S$ (the  source) and a set of geographic locations $\widehat{T}$ (the targets), determine the closest $N$ locations $t_i \in \widehat{T}$  to $S$. One can achieve this by computing the distance $d(t_i, S)$ from every target to the source but this is prohibitively slow even for a moderately large set of targets. A better solution is to make a bounding region $\cB$ around $S$ and compute the distance  $d(t_i,S)$ only for the $t_i \in \cB$. For well constructed $\cB$ one will find the $N$ closest points within $\cB$. 

There are numerous ways to store a set of geographic locations in memory or on disk but regardless of which specific method is utilized, the process of filtering $\widehat{T}$ down to the set of $t_i\in \cB$ is fast as it only involves a few numerical comparisons per target.


Geographic locations are typically given in the longitude-latitude coordinate system, so $\cB$ should be constructed from lines of equal latitude and lines of equal longitude. This results in $\cB$ which look roughly square close to the equator and a sort of curved triangle when near the poles. 


%%%%%%%%%%%%%%%%%%%%%%%%%%%%%%%%%%%%
\section{Constructing the Bounding Box}
%%%%%%%%%%%%%%%%%%%%%%%%%%%%%%%%%%%%

We denote $C_d$ the equidistant surface around $S$. it is the collection of points on the sphere which are at a distance $d$ from $S$.  To construct the bounding box $\cB$, we need to determine the minimum and maximum values of latitide and longitude attained by $C_d$.

%%%%%%%%%%%%%%%%%%%%%%%%%%%%%%%%%%%%
\subsection{Haversine Formula}
%%%%%%%%%%%%%%%%%%%%%%%%%%%%%%%%%%%%
 
The Haversine formula gives the distance between two points on the sphere in the latitude-longitude coordinate system: 
 \be
 \sin^2\Blp \frac{d}{2R}\Brp= \sin^2\Blp \frac{\vphi_1-\vphi_2}{2} \Brp+ \cos \vphi_1\cos \vphi_2 \sin^2\Blp \frac{\lam_1-\lam_2}{2}\Brp
 \ee
 or more conveniently
 \bea
\cos \frac{d}{R} &=& \sin \vphi_1 \sin \vphi_2 + \cos (\lam_2 - \lam_1)   \cos \vphi_1 \cos \vphi_2\,.
 \eea
where $R$ is the radius of the sphere.
%%%%%%%%%%%%%%%%%%%%%%%%%%%%%%%%%%%%
\subsection{Maximum Latitude}
The maximum latitude on $C_d$ is attained when the longitude is the same as the longitude of $S$.  When two longitudes are equal 
 \be
 \lam_2=\lam_1
 \ee
the Haversine formula gives
\be
\varphi_2-\varphi_1 = \frac{d}{R}
\ee 


%%%%%%%%%%%%%%%%%%%%%%%%%%%%%%%%%%%%
\subsection{Maxmimum Longitude}

%%%%%%%%%%%%%%%%%%%%%%%%%%%%%%%%%%%%
\subsubsection{Ellipse}

To warm up to the computation of  the point of maximum longitude, we first look at an ellipse in $\RR^2$:
\be
a^2 x^2 + b^2 y^2 = 1 
\ee
and look for the maximum value of $x$.

We solve for $x$
\bea
x(y) &= & \frac{1}{a}\sqrt{(1- b^2 y^2 )} 
\eea
and compute the variation of $x$ w.r.t. $y$ 
\bea
x'(y) = \frac{1}{a}\frac{2b^2y}{(1- b^2 y^2 )}\,.
\eea 
The extremum of $x$ are of course at the point where $x'(y)=0$ and this is at $y=0$, in particular on the $x$-axis.


%%%%%%%%%%%%%%%%%%%%%%%%%%%%%%%%%%%%
\subsubsection{Back to the Sphere}
Given a section of equal distance $d$ from the source $S=(\varphi_1, \lam_1)$ we want to know the maximum longitudinal distance, i.e. the maximum value of $|\lam_2-\lam_1|$. These will be the east and west boundaries of the bounding box.

We first fix $d$ and $(\varphi_1, \lam_1)$ then look for the extremum of $\lam_2$ as a function of $\vphi_2$. It is convenient to change coordinates to
\be
\Lam= \cos\blp\lam_1-\lam_2(\vphi_2)\brp\,,\qquad \Psi = \sin \vphi_2
\ee
then the equations to be solved are
\bea
\csc \vphi_1\cos \frac{d}{R}&=&\cot \vphi_1 \Lam \sqrt{1-\Psi^2}  +\Psi \\
\cot \vphi_1\Lam \Psi &=& \sqrt{1-\Psi^2}
\eea
The solution is 
\bea
\Lam &=& \sec(\vphi_1 ) \sqrt{\cos^2 \blp\frac{d}{R}\brp - \sin^2 (\vphi_1)} \\
\Psi &=&\sec \blp\frac{d}{R}\brp  \sin \vphi_1
\eea

There is a branch point  in $\Lam$ at one of the following points:
\be
{\rm b.p.} \quad \vphi_1 \pm \frac{d}{R} = \pm \pi/2
\ee
which is where the bounding section touches the north(south) pole. At this point the maximum longitude reached is $\lam_2-\lam_1 = \pi/2$ which is half way around the globe (though not necessarily at the equator).


%%%%%%%%%%%%%%%%%%%%%%%%%%%%%%%%%%%%
\subsubsection{Equal  Latitude}

One can also consider a related but inequivalent question, which is for the shortest distance between two points of equal latitude. This is slightly non-trivial since the great circle between two points does not follow lines of equal latitude unless both points are on the equator. Indeed in this interesting \href{http://janmatuschek.de/LatitudeLongitudeBoundingCoordinates}{blog post} the author realises the distinction between great circles and lines of constant latitude but mistakenly assumes that the maximum longitude is attained at the same latitude as the source $S$. 

To compute the distance between two points of equal latitude, we use the Haversine formula with
\be
\vphi_1=\vphi_2 = \vphi\,.
\ee
Then we find
\bea
d &=&  2R \sin^{-1} \Big| \cos \vphi   \sin \Blp \frac{\lam_1-\lam_2}{2}\Brp \Big|  \\
\Ra \quad |\lam_1 - \lam_2| &=& 2\left|\sin^{-1}\left[ \frac{\sin\left(\frac{d}{2R}\right)}{\cos(\varphi)}\right]\right|\,.
\eea
%The maximum longitudinal difference between two points is 
%\be
%|\lam_2-\lam_1|_{\rm max} = \pi
%\ee
%which occurs when
%\be
%\sin\left(\frac{d}{2R}\right)  = \cos(\varphi)\,.
%\ee
 





%%%%%%%%%%%%%%%%%%%%%%%%%%%%%%%%%%%%
\bibliographystyle{/Users/Halmagyi/Dropbox/utphys} 
\bibliography{/Users/Halmagyi/Dropbox/myrefs}
%%%%%%%%%%%%%%%%%%%%%%%%%%%%%%%%%%%%

%%%%%%%%%%%%%%%%%%%%%%%%%%%%%%%%%%%%
\end{document}
%%%%%%%%%%%%%%%%%%%%%%%%%%%%%%%%%%%%
%%%%%%%%%%%%%%%%%%%%%%%%%%%%%%%%%%%%
